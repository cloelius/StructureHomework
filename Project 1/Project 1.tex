% !TEX TS-program = pdflatex
% !TEX encoding = UTF-8 Unicode

% This is a simple template for a LaTeX document using the "article" class.
% See "book", "report", "letter" for other types of document.

\documentclass[11pt]{article} % use larger type; default would be 10pt
\usepackage{braket}
\usepackage[utf8]{inputenc} % set input encoding (not needed with XeLaTeX)

%%% Examples of Article customizations
% These packages are optional, depending whether you want the features they provide.
% See the LaTeX Companion or other references for full information.
\usepackage{amsmath}
%%% PAGE DIMENSIONS
\usepackage{geometry} % to change the page dimensions
\geometry{a4paper} % or letterpaper (US) or a5paper or....
% \geometry{margin=2in} % for example, change the margins to 2 inches all round
% \geometry{landscape} % set up the page for landscape
%   read geometry.pdf for detailed page layout information

\usepackage{graphicx} % support the \includegraphics command and options

% \usepackage[parfill]{parskip} % Activate to begin paragraphs with an empty line rather than an indent

%%% PACKAGES
\usepackage{booktabs} % for much better looking tables
\usepackage{array} % for better arrays (eg matrices) in maths
\usepackage{paralist} % very flexible & customisable lists (eg. enumerate/itemize, etc.)
\usepackage{verbatim} % adds environment for commenting out blocks of text & for better verbatim
\usepackage{subfig} % make it possible to include more than one captioned figure/table in a single float
% These packages are all incorporated in the memoir class to one degree or another...

%%% HEADERS & FOOTERS
\usepackage{fancyhdr} % This should be set AFTER setting up the page geometry
\pagestyle{fancy} % options: empty , plain , fancy
\renewcommand{\headrulewidth}{0pt} % customise the layout...
\lhead{}\chead{}\rhead{}
\lfoot{}\cfoot{\thepage}\rfoot{}

%%% SECTION TITLE APPEARANCE
\usepackage{sectsty}
\allsectionsfont{\sffamily\mdseries\upshape} % (See the fntguide.pdf for font help)
% (This matches ConTeXt defaults)

%%% ToC (table of contents) APPEARANCE
\usepackage[nottoc,notlof,notlot]{tocbibind} % Put the bibliography in the ToC
\usepackage[titles,subfigure]{tocloft} % Alter the style of the Table of Contents
\renewcommand{\cftsecfont}{\rmfamily\mdseries\upshape}
\renewcommand{\cftsecpagefont}{\rmfamily\mdseries\upshape} % No bold!

%%% END Article customizations

%%% The "real" document content comes below...

\title{Brief Article}
\author{The Author}
%\date{February 28, 2014} % Activate to display a given date or no date (if empty),
         % otherwise the current date is printed 

\begin{document}
\maketitle

\section{Quasispin operators}

We begin by defining the quasispin operators\\

\begin{equation}
J_+=\sum_p a_{p+}^\dagger a_{p-}
\end{equation}

\begin{equation}
J_-=\sum_p a_{p-}^\dagger a_{p+}
\end{equation}

\begin{equation}
J_z=\sum_{p\sigma}\sigma a_{p\sigma}^\dagger a_{p\sigma}
\end{equation}
	
\begin{equation}
J^2=J_+J_-+J_z^2-J_z
\end{equation}


We now set to show that these operators follow the canonical angular momenta commutator relations, which are as follows(setting hbar to 1):\\
\subsection{Angular Momenta Commutator Relatons}
\begin{equation}
\left[  J_+, J_-  \right]=2J_z 
\end{equation}

\begin{equation}
\left[J_+,J_z\right]=-J_+
\end{equation}
\begin{equation}
\left[J_-,J_z\right]=J_-
\end{equation}

\begin{equation}
\left[J^2, J_z\right]=0\end{equation}
\subsection{Proof of commutator relations}
\subsubsection{$\left[  J_+, J_-  \right]=2J_z $}
First we consider the commutator relation (5)

\begin{equation}
\left[J_+,J_-\right]=\sum_p a_{p+}^\dagger a_{p-} \sum_k a_{k-}^\dagger a_{k+}-\sum_k a_{k-}^\dagger a_{k+} \sum_p a_{p+}^\dagger a_{p-}   \end{equation}

We then immediately note that since for $k\neq p$ \\
\begin{equation}
\left[a_{p\sigma},a_{k\sigma'}\right]=0 
\end{equation}

It follows that all the terms where $p \neq k$ fall out of the commutator. Hence:
\begin{equation}
\left[J_+,J_-\right]=\sum_p a_{p+}^\dagger a_{p-} a_{p-}^\dagger a_{p+}-\sum_p a_{p-}^\dagger a_{p+}  a_{p+}^\dagger a_{p-}   \end{equation}

Now we note that because of the difference in spins, some of these terms commute and the sums can be merged and we can rewrite this as\\
\begin{equation}
\left[J_+,J_-\right]=\sum_p a_{p+}^\dagger a_{p+}a_{p-} a_{p-}^\dagger - a_{p-}^\dagger a_{p-}  a_{p+} a_{p+}^\dagger    \end{equation}

Finally we note that since \\
\begin{equation}
\left[a,a^\dagger\right]=1
\end{equation}

This thus means that with further rearangement\\

\begin{equation}
\left[J_+,J_-\right]=\sum_p a_{p+}^\dagger a_{p+} (a_{p-}^\dagger a_{p-}+1) - (a_{p+}^\dagger a_{p+} +1) a_{p-}^\dagger a_{p-}      \end{equation}


And finally simplifying this is\\

\begin{equation}
\left[J_+,J_-\right]=\sum_p a_{p+}^\dagger a_{p+}  - a_{p-}^\dagger a_{p-}      \end{equation}

Which we note is just\\

\begin{equation}
\left[J_+,J_-\right]=\sum_{p\sigma} a_{p\sigma}^\dagger a_{p\sigma} =2J_z     \end{equation}

And this is as expected in agreement with the relationship (5) above.\\


\subsubsection{$\left[J_+,J_z\right]=-J_+$}

Again we write this out in full using the creation and anhilation operators. \\


\begin{equation}
\left[J_+,J_z\right]=\frac{1}{2} \left( \sum_k a_{k+}^\dagger a_{k-} \sum_{p\sigma}\sigma a_{p\sigma}^\dagger a_{p\sigma}- \sum_{p\sigma}\sigma a_{p\sigma}^\dagger a_{p\sigma}\sum_k a_{k+}^\dagger a_{k-}\right)
\end{equation}

From this it follows that if $k \neq p$ it follows that those terms cancel and similarly for the sigmas. \\

\begin{equation}
\left[J_+,J_z\right]=\frac{1}{2}\sum_{p\sigma} \sigma \left( a_{p+}^\dagger a_{p-}  a_{p\sigma}^\dagger a_{p\sigma}-  a_{p\sigma}^\dagger a_{p\sigma} a_{p+}^\dagger a_{p-}\right)
\end{equation}

We then expand this sum over sigma as\\

\begin{equation}
\left[J_+,J_z\right]=\frac{1}{2}\sum_{p}  \left( a_{p+}^\dagger a_{p-}  a_{p+}^\dagger a_{p+}-  a_{p+}^\dagger a_{p+} a_{p+}^\dagger a_{p-}  - a_{p+}^\dagger a_{p-}  a_{p-}^\dagger a_{p-}+  a_{p-}^\dagger a_{p-} a_{p+}^\dagger a_{p-} \right)
\end{equation}


Again taking advantage of the creation/annhilation operators' commutation relations it follows that\\


\begin{equation}
\left[J_+,J_z\right]=\frac{1}{2}\sum_{p}  \left( a_{p+}^\dagger  a_{p+}^\dagger a_{p+} a_{p-}-  a_{p+}^\dagger (a_{p+}^\dagger a_{p+}+1) a_{p-}  - a_{p+}^\dagger  (a_{p-}^\dagger a_{p-}+1)  a_{p-}+   a_{p+}^\dagger a_{p-}^\dagger a_{p-} a_{p-} \right)
\end{equation}

And finally cancelling like terms\\


\begin{equation}
\left[J_+,J_z\right]=-\sum_{p} a_{p+}^\dagger a_{p-}=-J_+
\end{equation}


\subsubsection{$\left[J_-,J_z\right]=J_-$}

This is fundamentally the same as the previous argument and so I skip most of the explanations:\\


\begin{equation}
\left[J_-,J_z\right]=\frac{1}{2} \left( \sum_k a_{k-}^\dagger a_{k+} \sum_{p\sigma}\sigma a_{p\sigma}^\dagger a_{p\sigma}- \sum_{p\sigma}\sigma a_{p\sigma}^\dagger a_{p\sigma}\sum_k a_{k-}^\dagger a_{k+}\right)
\end{equation}


\begin{equation}
\left[J_-,J_z\right]=\frac{1}{2}\sum_{p\sigma} \sigma \left( a_{p-}^\dagger a_{p+}  a_{p\sigma}^\dagger a_{p\sigma}-  a_{p\sigma}^\dagger a_{p\sigma} a_{p-}^\dagger a_{p+}\right)
\end{equation}



\begin{equation}
\left[J_-,J_z\right]=\frac{1}{2}\sum_{p}  \left( a_{p-}^\dagger a_{p+}  a_{p+}^\dagger a_{p+}-  a_{p+}^\dagger a_{p+} a_{p-}^\dagger a_{p+}  - a_{p-}^\dagger a_{p+}  a_{p-}^\dagger a_{p-}+  a_{p-}^\dagger a_{p-} a_{p-}^\dagger a_{p+} \right)
\end{equation}


Again taking advantage of the creation/annhilation operators' commutation relations it follows that\\


\begin{equation}
\left[J_-,J_z\right]=\frac{1}{2}\sum_{p}  \left( a_{p+}^\dagger  a_{p+}^\dagger a_{p-} a_{p+}-  a_{p+}^\dagger (a_{p+}^\dagger a_{p+}-1) a_{p-}  - a_{p-}^\dagger  (a_{p-}^\dagger a_{p-}-1)  a_{p+}+   a_{p-}^\dagger a_{p-}^\dagger a_{p-} a_{p+} \right)
\end{equation}

And finally cancelling like terms\\


\begin{equation}
\left[J_-,J_z\right]=\sum_{p} a_{p-}^\dagger a_{p}=J_-
\end{equation}

\subsubsection{$\left[J^2,J_z\right]=0$}

Here we take advantage of commutator algebra. Noting that\\

\begin{equation}
J^2=J_+J_-+J_z^2-J_z
\end{equation}

It is immediately clear that $J_z$ commutes with the second and third terms.\\

Then\\

\begin{equation}
\left[J^2,J_z\right]=\left[J_+J_-,J_z \right]=J_+\left[J_-,J_z\right]+\left[J_+,J_z\right]J_-=J_+J_- -J_+J_-=0
\end{equation}


\section{Hamiltonian in Quasi Spin Form}

We need to rewrite the hamiltonian operator in terms of these quasispin operators. 

\subsection{$H_0=\frac{1}{2}\epsilon \sum_{\sigma,p} \sigma a_{\sigma,p}^\dagger  a_{\sigma,p}$}

We see immediately that this is just\\

\begin{equation}
H_0=\epsilon \hat{J_z}
\end{equation}

Which of course fits with the anticpated physical phenomenon.

\subsection{$H_1=\frac{V}{2}\sum_{\sigma,p,p'} a_{\sigma,p}^\dagger a_{\sigma,p'}^\dagger a_{-\sigma,p'}a_{-\sigma,p}$}

This derivation is fairly straightforward, as due to the fact that the anhiliation and creation operators have different energy levels(i.e. the creation operators are all at energy $\sigma$ and the annhilation at energy -$\sigma$), it follows that\\

\begin{equation}
H_1=\frac{1}{2}V \sum_{\sigma,p,p'} a_{\sigma p}^\dagger a_{-\sigma, p} a_{\sigma p'}^\dagger a_{-\sigma, p} 
\end{equation}



This can then immediately be decomposed into a product over the two different $\sigma$ states. Thus these become\\

\begin{equation} 
\frac{V}{2}\left(\sum_p a_{+ p}^\dagger a_{-, p} \sum_{p'} a_{+ p'}^\dagger a_{-, p'}  +\sum_p a_{- p}^\dagger a_{+, p} \sum_{p'} a_{- p'}^\dagger a_{+, p'}\right)
\end{equation}

And it is then clear that this is just\\

\begin{equation}
\frac{V}{2}\left((J^+)^2+(J^-)^+\right)
\end{equation}

\subsection{$H_2=\frac{W}{2}\sum_{\sigma,p,p'}a_{\sigma,p}^\dagger a_{-\sigma,p'}^\dagger a_{\sigma,p'} a_{-\sigma,p}$}


This derivation is similar to the above except in that if p=p' there is  need to make use of a commutation relation:\\

\begin{equation}
H_2=\frac{W}{2}\sum_{\sigma,p,p'}a_{\sigma,p}^\dagger \left(a_{-\sigma,p}
 a_{-\sigma,p'}^\dagger -\delta_{p,p'}\right) a_{\sigma,p'} 
\end{equation}

This can thus be written as
\begin{equation}
 \frac{W}{2}\sum_{\sigma,p,p'}a_{\sigma,p}^\dagger a_{-\sigma,p}
 a_{-\sigma,p'}^\dagger a_{\sigma,p'}  -\sum_{\sigma,p}a_{\sigma,p}^\dagger a_{\sigma,p}
\end{equation}
It is clear that the second term is identical to
\begin{equation}
\hat{N}=\sum_{\sigma,p}a_{\sigma,p}^\dagger a_{\sigma,p}
\end{equation} 

It is furthermore clear that the first term can be decomposed as in the previous section so that
\begin{equation}
 \sum_\sigma \left(\sum_{p}a_{\sigma,p}^\dagger a_{-\sigma,p}
 \sum_{p'}a_{-\sigma,p'}^\dagger a_{\sigma,p'} \right)
\end{equation}

And we see this term is just\\

\begin{equation}
J_+J_-+J_-J_+
\end{equation}

Finally putting all these terms together we have that 

\begin{equation}
H_2=\frac{W}{2}\left( J_+J_- + J_-J_+ -N\right)
\end{equation}

Finally taking advantage of the commutation relations in part one and the definition of $J^2$ it follows that\\

\begin{equation}
H_2=W\left(J^2-J_z^2-\frac{N}{2}\right)
\end{equation}


\section{$J^2$ commutes with H}

To prove $J^2$ commutes with H we show it commutes with each component of H.
\subsection{$H_0$}
 Having already shown that $\left[J^2,J_z\right]=0$ it is trivially clear that\\

\begin{equation}
\left[J^2,H_0\right]=\left[J^2,\epsilon J_z \right]=\epsilon \left[J^2,J_z \right]=0
\end{equation}
\subsection{$H_1$}
Next we show that \\

\begin{equation}
\left[J^2,H_1\right]=0
\end{equation}

We can show this by showing instead that\\

\begin{equation}
\left[J^2,J_+\right]=0
\end{equation}

and\\
\begin{equation}
\left[J^2,J_-\right]=0
\end{equation}

We prove these by takng advantage of the fact that

\begin{equation}
J^2=J_+J_-+J_z^2-J_z
\end{equation}
\subsubsection{$\left[J^2,J_+\right]=0$}
From this it follows that\\
\begin{equation}
\left[J^2,J_+\right]=\left[J_+J_-,J_+\right]+\left[J_z^2,J_+\right]-\left[J_z,J_+\right]
\end{equation}

By taking advantage of our previous commutation relations and basic commutator algebra we have\\

\begin{equation}
\left[J^2,J_+\right]=J_+\left[J_-,J_+\right]+J_z\left[J_z,J_+\right]+\left[J_z,J_+\right]J_z-J_+\end{equation}


And with further simplification\\

\begin{equation}
\left[J^2,J_+\right]=-2J_+J_z+J_zJ_+ + J_+ J_z-J_+=2J_+J_z-2J_+J_z+J_+-J_+=0\end{equation}

\subsubsection{$\left[J^2,J_-\right]=0$}
We can then do the same for the $J_-$ case.

\begin{equation}
\left[J^2,J_-\right]=\left[J_+J_-,J_-\right]+\left[J_z^2,J_-\right]-\left[J_z,J_-\right]
\end{equation}


\begin{equation}
\left[J^2,J_-\right]=\left[J_+,J_-\right]J_-+J_z\left[J_z,J_-\right]+\left[J_z,J_-\right]J_z+J_-\end{equation}


And with further simplification\\

\begin{equation}
\left[J^2,J_+\right]=2J_zJ_-+J_zJ_- + J_- J_z+J_-=2J_zJ_- -2J_zJ_-+J_--J_-=0\end{equation}

From this it immediately follows that 
\begin{equation}
\left[J^2,H_1\right]=\left[J^2, \frac{V}{2}\left((J^+)^2+(J^-)^2\right)\right]=0
\end{equation}

\subsection{$\left[J^2,H_2\right]=0$}

We prove this is also 0 by noting that as shown below the $H_2$ term has a term $J^2$ which trivially commutes with itself, and a term in powers of $J_z$ which we have already shown commutes. Hence we only need to show that the remaining term commutes, or equivalently that $\left[J^2,N\right]=0$\\
\begin{equation}
H_2=W\left(J^2-J_z^2-\frac{N}{2}\right)
\end{equation}

We prove this by first finding the commutators\\

\begin{equation}
\left[J_+,N\right], \left[J_-,N\right]
\end{equation}
\subsubsection{$\left[J_-,N\right]$}
We can then have\\

\begin{equation}
\left[J_-,N\right]=\left[\sum_p a_{p-}^\dagger a_{p+}, \sum_p a_{p+}^\dagger a_{p+} +\sum_p a_{p-}^\dagger a_{p-}\right]
\end{equation}

Noting that the only non zero terms of the commutator will be those when the p terms are identical we thus have

\begin{equation}
\left[J_-,N\right]=\sum_p\left[ a_{p-}^\dagger a_{p+},  a_{p+}^\dagger a_{p+} + a_{p-}^\dagger a_{p-}\right]
\end{equation}
Separating out the sums we have
\begin{equation}
\left[J_-,N\right]=\sum_p\left[ a_{p-}^\dagger a_{p+},  a_{p+}^\dagger a_{p+} + a_{p-}^\dagger a_{p-}\right]
\end{equation}

\begin{equation}
\left[J_-,N\right]=\sum_p\left[ a_{p-}^\dagger a_{p+},  a_{p+}^\dagger a_{p+} \right]+\left[ a_{p-}^\dagger a_{p+},  a_{p-}^\dagger a_{p-}\right]
\end{equation}

Simplifying further we have\\

\begin{equation}
\left[J_-,N\right]=\sum_p\left( a_{p-}^\dagger\left[ a_{p+},  a_{p+}^\dagger \right] a_{p+}+ a_{p-}^\dagger\left[ a_{p-}^\dagger,  a_{p-} \right] a_{p+}\right)
\end{equation}

Taking the commutators we see that\

\begin{equation}
\left[J_-,N\right]=\sum_p a_{p-}^\dagger a_{p+} -\sum_p a_{p-}^\dagger a_{p+}
\end{equation}

And we see this is equal to\\
\begin{equation}
\left[J_-,N\right]=J_- -J_-=0
\end{equation}

\subsubsection{$\left[J_+,N\right]$}
We now do the same for $J_+$\\

\begin{equation}
\left[J_+,N\right]=\left[\sum_p a_{p+}^\dagger a_{p-}, \sum_p a_{p+}^\dagger a_{p+} +\sum_p a_{p-}^\dagger a_{p-}\right]
\end{equation}

Noting that the only non zero terms of the commutator will be those when the p terms are identical we thus have

\begin{equation}
\left[J_+,N\right]=\sum_p\left[ a_{p+}^\dagger a_{p-},  a_{p+}^\dagger a_{p+} + a_{p-}^\dagger a_{p-}\right]
\end{equation}
Separating out the sums we have


\begin{equation}
\left[J_+,N\right]=\sum_p\left[ a_{p+}^\dagger a_{p-},  a_{p+}^\dagger a_{p+} \right]+\left[ a_{p+}^\dagger a_{p-},  a_{p-}^\dagger a_{p-}\right]
\end{equation}

Simplifying further we have\\

\begin{equation}
\left[J_+,N\right]=\sum_p\left( a_{p+}^\dagger\left[ a_{p+}^\dagger,  a_{p+} \right] a_{p-}+ a_{p+}^\dagger\left[ a_{p-},  a_{p-}^\dagger \right] a_{p-}\right)
\end{equation}

Taking the commutators we see that\

\begin{equation}
\left[J_+,N\right]=-\sum_p a_{p+}^\dagger a_{p-} -\sum_p a_{p+}^\dagger a_{p-}
\end{equation}

And we see this is equal to\\
\begin{equation}
\left[J_+,N\right]=J_+ -J_+=0
\end{equation}

\subsubsection{$\left[J^2,N\right]$}

We note finally that if we define an operator 
\begin{equation}n_{\sigma}=\sum_p a_{p\sigma}^\dagger a_{p\sigma}\end{equation}

It follows that trivially \\

\begin{equation}
\left[n_{\sigma}.n_{\sigma'}\right]=0
\end{equation}

It thereby follows that \\

\begin{equation}
J_z=\frac{1}{2}\left(n_+ - n_-\right)
\end{equation}

and\\

\begin{equation}
N=n_++n_-
\end{equation}

Now from this it therefore follows that N and $J_z$ commute trivially.\\

Hence \\

\begin{equation}
\left[J^2,N\right]=0
\end{equation}

\section{Constructing the $J=2$ states}
\subsection{$J_z=-1$}
We begin by considering that the $J=-2$ state is\\

\begin{equation} \left(\prod  _p a_{-p}^\dagger\right) \ket{ 0}\end{equation}
\subsection{$J_z=-1$}
We then apply the $J_+$ operator knowing that\\

\begin{equation}
J_+ \ket{J,J_z}=\sqrt{J(J+1)-J_z(J_z+1)}\ket{J,J_z+1}\end{equation}

We can then directly apply this as\\

\begin{equation} 
J_+ \ket{2,-2}=\sqrt{6-2}\ket{2,-1}=\sum_p a_{p+}^\dagger a_{p-}\left(\prod_p a_{p-}^\dagger\right) \ket{ 0} \end{equation}


We see that this last term is such that if the anhiliation term comes first it will kill of the term. Hence we get a delta function in ps over the sum, leading to\\

\begin{equation}
\sqrt{4}\ket{2,-1}=\sum_p a_{p+}^\dagger a_{p-} a_{p-}^\dagger \ket{ 0} =\sum_p a_{p+}^\dagger \prod_{p'} \left(\left( a_{p'-}^\dagger a_{p-}+1\right)\delta_{p,p'}+a_{p'-}^\dagger(1-\delta_{p,p'}) \right)  \ket{ 0} \end{equation}

Now the term in the product on the far left clearly goes to zero so that this becomes\\

\begin{equation}
\sum_p a_{p+}^\dagger \prod_{p' \neq p} a_{p'-}^\dagger
\end{equation}

We thus have\\
\begin{equation}
\ket{2,-1}=\frac{1}{2} \left(\sum_p a_{p+}^\dagger \prod_{p' \neq p} a_{p'-}^\dagger\right)\end{equation}


It is clear that this is just the normalized sum of all possible permutations of levels where one is $\sigma=+$ and all the others are -1.


\subsection{$J_z=0$}
We can then apply the $J_+$ operator again.\\

\begin{equation}
J_+ \ket{2,-1}=\sqrt{6} \ket{2,0}=\frac{1}{2} \left(\sum_{p''} a_{p''+}^\dagger a_{p'' -} \sum_p a_{p+}^\dagger \prod_{p' \neq p} a_{p'-}^\dagger\right)
\end{equation}

We see clearly that this follows a similar pattern to the previous, with the only major difference being in that since the sum is over both p and p', that there is a factor of two when the delta functions are considered. We thereby wind up with\\

\begin{equation}
\ket{2,0}=\frac{1}{\sqrt{6}}\left(\sum_{p,p'\neq p} a_{p+}^\dagger a_{p'+}^\dagger \prod_{p'''\neq p\neq p'} a_{p''-}^\dagger \right)
\end{equation}

Which is to say this is just all the possibilites where all four degeneraties are occupied and two are up and two down normalized.\\

\subsection{$J_z=1$}

We here note that via the symmetry of the hamiltonian under $\epsilon \rightarrow -\epsilon$ and $\sigma \rightarrow -\sigma$ it follows that\\

\begin{equation} \ket{2, 1} =\ket{2,-1} |_{\sigma \rightarrow -\sigma}=\frac{1}{2} \left(\sum_p a_{p-}^\dagger \prod_{p' \neq p} a_{p'+}^\dagger\right)\end{equation}


It is clear that this is just the normalized sum of all possible permutations of levels where one is $\sigma=-$ and all the others are +.

\subsection{$J_z=2$}

The same argument as previous holds and so we trivially have\\

\begin{equation}
\ket{2,2}=\sum_p a_{p+}^\dagger
\end{equation}


\section{Hamiltonian in 5d space}

We can construct this Hamiltonian matrix in general by considering the application of $J_+$, $J_-$, $J_z$ and $N$ on the eigenstates $\ket{J,J_z}$.(so long as the hamiltonian is made up of these operators).\\


In this case we can see trivially from the way that the operators $J_+$, $J_-$ and $J_z$ operate that their matrix representations in general ought to be, where we take the basis to be \\
\begin{equation}
\vec{J}=\begin{pmatrix} \ket{J,J_z=J} \\ \ket{J,J_z=J-1} \\ \vdots \\ \ket{J,J_z=-J} \end{pmatrix}\end{equation}
\begin{equation}
J_-=\begin{pmatrix} 0 & 0 & 0 & \cdots & 0 \\
\sqrt{J(J+1)-(J-1)(J)} & 0 & 0 & \cdots & 0 \\
0& \sqrt{J(J+1)-(J-2)(J-1)}  & 0 & \cdots & 0 \\
\vdots & \ddots  & \ddots  & \cdots & \vdots\\
0 & 0 & 0 & \sqrt{J(J+1)-J(J-1)}& 0 

 \end{pmatrix}
\end{equation}  


Similarly for the $J_-$ we have\\

\begin{equation}
J_+=\begin{pmatrix} 0 & \sqrt{J(J+1)-J(J-1)} & 0 & \cdots & 0 \\
0 & 0 & \sqrt{J(J+1)-(J-1)(J-2)} & \cdots & 0 \\
\vdots & \ddots  & \ddots  & \cdots & \vdots\\
0 & 0 & 0 & 0 & \sqrt{J(J+1)-(J-1)(J)}\\
0 & 0 & 0 & 0& 0 

 \end{pmatrix}
\end{equation}  


And for $J_z$ we have trivially that\\

\begin{equation}
J_z=\begin{pmatrix} J & 0 & 0 & \cdots & 0 \\
0 & J-1  & 0 & \cdots & 0 \\
\vdots & \ddots  & \ddots  & \cdots & \vdots\\
0 & 0 & 0 & -J+1 & 0\\
0 & 0 & 0 & 0& -J 
 \end{pmatrix}
\end{equation}
Now finally we note that $N$ returns the original state with an eigenvalue of how many particles are in the state. It's worth noting then that we can define $J$ states for each of the possible numbers of particles N=(0,1,2,3,4) which map to $J=(0,\frac{1}{2},1,\frac{3}{2},2)$

  From this it immediately follows that\\

\begin{equation}
N=2\begin{pmatrix} 
J & 0 & 0 & \cdots & 0 \\
0 & J  & 0 & \cdots & 0 \\
\vdots & \ddots  & \ddots  & \cdots & \vdots\\
0 & 0 & 0 & J & 0\\
0 & 0 & 0 & 0& J \end{pmatrix}
\end{equation}

\section{Hamiltonian}

The Hamiltonian itself is thus easily constructed out of the J matricies by substitution, leading to:\\

\begin{equation}\resizebox{.9\hsize}{!}{$
H=H_0+H_1+H_2=\epsilon \begin{pmatrix}2 & 0 & 0 & 0 & 0 \\
0 & 1 & 0 & 0 & 0 \\
0 & 0 & 0 & 0 & 0 \\
0 & 0 & 0 & -1 & 0 \\
0 & 0 & 0 & 0 & -2
\end{pmatrix}+V \begin{pmatrix}0 & 0 & 2.45 & 0 & 0 \\
0 & 0 & 0 & 3 & 0 \\
2.45 & 0 & 0 & 0 & 2.45 \\
0 & 3 & 0 & 0 & 0 \\
0 & 0 & 2.45 & 0 & 0 \end{pmatrix}+W\begin{pmatrix}0 & 0 & 0 & 0 & 0 \\
0 & 3 & 0 & 0 & 0 \\
0 & 0 & 4 & 0 & 0 \\
0 & 0 & 0 & 3 & 0 \\
0 & 0 & 0 & 0 & 0\end{pmatrix}$}
\end{equation}
 
Hence we have that\\

\begin{equation}
H=\begin{pmatrix} 2\epsilon & 0 &2.45 V & 0 & 0\\
			0 & 3 W + \epsilon & 0& 3 V & 0\\
			2.45 V & 0 & 4W & 0 & 2.45 V\\
			0 & 3 V & 0 & 3 W-\epsilon & 0 \\
			0 & 0 & 2.45 V & 0 &-2\epsilon  \end{pmatrix} \end{equation}


\section{Eigenvalues/Eigenvectors of the Hamiltonian}

Now we consider both the case of $\epsilon=2, V=-\frac{1}{3}, W=-\frac{1}{4}$ and $\epsilon=2, V=-\frac{4}{3}, W=-1$. This is done using the script attached in the appendix.\\


\subsection{$\epsilon=2, V=-\frac{1}{3}, W=-\frac{1}{4}$}

Here we have the matrix of eigenvectors is 
\begin{equation} \mathbf{\tilde{\lambda}}=\begin{vmatrix} 0.99 & 0.16 & 0.03 & 0.00 & 0.00 \\
0.00 & 0.00 & 0.00 & -0.23 & -0.97 \\
-0.16 & 0.95 & 0.25 & 0.00 & 0.00 \\
0.00 & 0.00 & 0.00 & -0.97 & 0.23 \\
0.02 & -0.25 & 0.97 & 0.00 & 0.00\end{vmatrix}
\end{equation}

These correspond to the eigenvalues
\begin{equation}
\lambda=\begin{pmatrix} 4.132 & -0.919 & -4.213 & -2.986 & 1.486\end{pmatrix} \end{equation}


We can see that in this case the ground state is that with energy  -4.213, which is the state\\

\begin{equation}
\psi=0.97 \ket{2,-2}+0.25 \ket{2,0}+0.03 \ket{2,2} \end{equation}

To describe why this is, we first note that in this case the $H_0$ and $H_2$ both act on the same eigenstates(which would be the unmixed $J_m$ states). What's more the $H_1$ term that does not have the same eigenvectors can only increase or decrease the $J_m$ states by 2, which explains why the eigenvectors for the total $H$ matrix are easily divided into odd and even catagories. In this case, the relative strength of the $H_1$ matrix is weak, and so the ground state is that where $J_z$ is lowest, since the $H_0$ still dominates, and this strongly favors $J_z=-J$ states. 

\subsection{ $\epsilon=2, V=-\frac{4}{3}, W=-1$}

Now we consider the same for the situation where the relative importance of $H_1$ and $H_2$ are important. We can thus expect a change in the eigenvalues due to $H_2$ and a change in the eigenvectors(and eigenvalues) due to $H_1$. As such we ought to expect far more mixing of the states. This is confirmed when we look at the matrix of eigenvectors\\

\begin{equation}\mathbf{\tilde{\lambda}}=\begin{vmatrix}-0.92 & -0.33 & 0.21 & 0.00 & 0.00 \\
0.00 & 0.00 & 0.00 & -0.53 & -0.85 \\
0.37 & -0.57 & 0.74 & 0.00 & 0.00 \\
0.00 & 0.00 & 0.00 & -0.85 & 0.53 \\
-0.13 & 0.76 & 0.64 & 0.00 & 0.00\end{vmatrix}\end{equation}

Which correspond to eigenvalues\\
\begin{equation}
\lambda=\begin{pmatrix} 5.31 & -1.56 & -7.751 & -7.472 &1.472 \end{pmatrix} \end{equation} 

Hence we see in this case that the ground state has energy $-7.751$ and corresponds to\\
\begin{equation} \psi=0.21 \ket{2,2} + 0.74 \ket{2,0}+0.64 \ket{2,-2} \end{equation}

This is a substantial admixture of states, and in fact has less of the $J_z=-2$ state than the $J_z=0$ state, but because of the mixing term is the ground state. We can see the importance of the ixing term by considering that the odd term \\
\begin{equation}
\psi=-0.53 \ket{2,1} -0.85 \ket{2,-1} \end{equation}

Has an energy of -7.472 which is very close to the ground state, and is clearly mostly dominated by the mixing term, though it also has less of the $J_z=2$ term to factor in. That is, there is a competition between admixtures of even states that include both the high energy $\ket{2,2}$ state and the $\ket{2,-2}$ state which also has significant mixing, and the odd states that have less of the $H_0$ term and so less of the high energy states from it, but also consequently have somewhat less mixing. \\

\subsection{Pure $H_2$}
For a final comparison I note that if $W=\epsilon=0$ and $V=1$ it follows that the matrix of eigenvectors is\\
\begin{equation}\mathbf{\tilde{\lambda}}=\begin{vmatrix}0.50 & -0.71 & 0.50 & 0.00 & 0.00 \\
-0.00 & 0.00 & 0.00 & -0.71 & 0.71 \\
0.71 & -0.00 & -0.71 & 0.00 & 0.00 \\
-0.00 & 0.00 & 0.00 & 0.71 & 0.71 \\
0.50 & 0.71 & 0.50 & 0.00 & 0.00 \end{vmatrix} \end{equation}

With corresponding eigenvalues\\

\begin{equation}
\lambda=\begin{pmatrix} -3.464 & 0 & 3.464 & 3 & -3 \end{pmatrix}\end{equation}

We see that this again splits the eigenvectors into even and odd solutions, and indeed the odd solutions have lower absolute values for energy than the even cases, except that where there is no admixture of the $J_z=0$ term which has a balanced energy of 0. This fits what was observed in the Hamiltonian before.\\


\section{Unitary Transformation}

We now note that we can make a new single particle state out of the lipkin model by summing over sum mixture of Lipkin model particles\\

\begin{equation} \ket{\phi_{\alpha,\sigma}}=\sum_{\sigma=\pm 1} C_{\alpha \sigma} \ket{\mu_{\sigma,p}}\end{equation}

We note further that these states will be orthonormal if we assume that $\mathbf{\tilde{C}}$ is unitary. This is because it would then follow that\\

\begin{equation}
\bra{\phi_{\alpha',p'}}\ket{\phi_{\alpha,p}}=\left(\sum_{\sigma'=\pm 1} C_{\alpha' \sigma'}^*\bra{\mu_{\sigma',p'}}\right)\left(\sum_{\sigma=\pm 1} C_{\alpha \sigma}\ket{\mu_{\sigma,p}}\right)\end{equation}

We realize that by the orthonormality of the Lipkin states this means that\\
\begin{equation}
\bra{\phi_{\alpha',p'}}\ket{\phi_{\alpha,p}}= \sum_{\sigma=\pm 1} C_{\alpha' \sigma}^* C_{\alpha \sigma} \bra{\mu_{\sigma,p'}} \ket{\mu_{\sigma,p}}=\sum_{\sigma=\pm 1} C_{\alpha' \sigma}^* C_{\alpha \sigma} \delta_{p,p'}\end{equation}




\end{document}

