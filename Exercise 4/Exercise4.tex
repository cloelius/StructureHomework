% !TEX TS-program = pdflatex
% !TEX encoding = UTF-8 Unicode

% This is a simple template for a LaTeX document using the "article" class.
% See "book", "report", "letter" for other types of document.

\documentclass[11pt]{article} % use larger type; default would be 10pt

\usepackage[utf8]{inputenc} % set input encoding (not needed with XeLaTeX)

%%% Examples of Article customizations
% These packages are optional, depending whether you want the features they provide.
% See the LaTeX Companion or other references for full information.

%%% PAGE DIMENSIONS
\usepackage{geometry} % to change the page dimensions
\geometry{a4paper} % or letterpaper (US) or a5paper or....
% \geometry{margin=2in} % for example, change the margins to 2 inches all round
% \geometry{landscape} % set up the page for landscape
%   read geometry.pdf for detailed page layout information

\usepackage{graphicx} % support the \includegraphics command and options

% \usepackage[parfill]{parskip} % Activate to begin paragraphs with an empty line rather than an indent

%%% PACKAGES
\usepackage{booktabs} % for much better looking tables
\usepackage{array} % for better arrays (eg matrices) in maths
\usepackage{paralist} % very flexible & customisable lists (eg. enumerate/itemize, etc.)
\usepackage{verbatim} % adds environment for commenting out blocks of text & for better verbatim
\usepackage{subfig} % make it possible to include more than one captioned figure/table in a single float
% These packages are all incorporated in the memoir class to one degree or another...

\usepackage{braket}
\usepackage{listings}
\usepackage{color}

\definecolor{mygreen}{rgb}{0,0.6,0}
\definecolor{mygray}{rgb}{0.5,0.5,0.5}
\definecolor{mymauve}{rgb}{0.58,0,0.82}

\lstset{ %
  backgroundcolor=\color{white},   % choose the background color; you must add \usepackage{color} or \usepackage{xcolor}
  basicstyle=\footnotesize,        % the size of the fonts that are used for the code
  breakatwhitespace=false,         % sets if automatic breaks should only happen at whitespace
  breaklines=true,                 % sets automatic line breaking
  captionpos=b,                    % sets the caption-position to bottom
  commentstyle=\color{mygreen},    % comment style
  deletekeywords={...},            % if you want to delete keywords from the given language
  escapeinside={\%*}{*)},          % if you want to add LaTeX within your code
  extendedchars=true,              % lets you use non-ASCII characters; for 8-bits encodings only, does not work with UTF-8
  frame=single,                    % adds a frame around the code
  keepspaces=true,                 % keeps spaces in text, useful for keeping indentation of code (possibly needs columns=flexible)
  keywordstyle=\color{blue},       % keyword style
  language=Octave,                 % the language of the code
  morekeywords={*,...},            % if you want to add more keywords to the set
  numbers=left,                    % where to put the line-numbers; possible values are (none, left, right)
  numbersep=5pt,                   % how far the line-numbers are from the code
  numberstyle=\tiny\color{mygray}, % the style that is used for the line-numbers
  rulecolor=\color{black},         % if not set, the frame-color may be changed on line-breaks within not-black text (e.g. comments (green here))
  showspaces=false,                % show spaces everywhere adding particular underscores; it overrides 'showstringspaces'
  showstringspaces=false,          % underline spaces within strings only
  showtabs=false,                  % show tabs within strings adding particular underscores
  stepnumber=2,                    % the step between two line-numbers. If it's 1, each line will be numbered
  stringstyle=\color{mymauve},     % string literal style
  tabsize=2,                       % sets default tabsize to 2 spaces
  title=\lstname                   % show the filename of files included with \lstinputlisting; also try caption instead of title
}
%%% HEADERS & FOOTERS
\usepackage{fancyhdr} % This should be set AFTER setting up the page geometry
\pagestyle{fancy} % options: empty , plain , fancy
\renewcommand{\headrulewidth}{0pt} % customise the layout...
\lhead{}\chead{}\rhead{}
\lfoot{}\cfoot{\thepage}\rfoot{}
\usepackage{amsmath}
%%% SECTION TITLE APPEARANCE
\usepackage{sectsty}
\allsectionsfont{\sffamily\mdseries\upshape} % (See the fntguide.pdf for font help)
% (This matches ConTeXt defaults)
\usepackage{isomath}
%%% ToC (table of contents) APPEARANCE
\usepackage[nottoc,notlof,notlot]{tocbibind} % Put the bibliography in the ToC
\usepackage[titles,subfigure]{tocloft} % Alter the style of the Table of Contents
\renewcommand{\cftsecfont}{\rmfamily\mdseries\upshape}
\renewcommand{\cftsecpagefont}{\rmfamily\mdseries\upshape} % No bold!

%%% END Article customizations

%%% The "real" document content comes below...

\title{Exercise 4}
\author{Charles Loelius}
%\date{} % Activate to display a given date or no date (if empty),
         % otherwise the current date is printed 

\begin{document}
\maketitle

\section{Slater Determinant Transforms}

The objective here is to show that taking a slater determinant into a new orthonormal basis will still allow it to be written as a Slater determinant.\\


To begin, we assume we have some wavefunctions and particles $\psi_n$ and $x_n$, which can be put into the standard format as:\\

\begin{equation}
\matrixsym{\Psi}=\left( \begin{matrix} \psi_1(x_1) & \ldots & \psi_1(x_n) \\ \vdots & \ddots & \vdots \\ \psi_n(x_1) & \ldots & \psi_n(x_n)  \end{matrix}\right)
\end{equation}

Where trivially we have that the wavefunction is the determinant of this matrix $\matrixsym{\Psi}$.\\


Now let us consider a conversion to another orthonormal basis. We then know that this means that there is a relationship\\

\begin{equation}
\phi_a=\Sigma_i C_{a, i}\psi_i
\end{equation}

We then realize that this can itself be represented as a matrix of the form\\

\begin{equation}
\vec{\phi}=\matrixsym{C}\vec{\psi}
\end{equation}

Where each of the vectors $\vec{\phi}$ and $\vec{\psi}$ are the orthnormal bases.

So with this we can then note that if we wish to write out a particular $\psi_a$ in terms of $\phi$, it follows that\\

\begin{equation}
\matrixsym{C^{-1}}\vec{\phi}=\vec{\psi}
\end{equation}

We then recognize that each column of the matrix $\matrixsym{\Psi}$ is just the basis vector with a different $x_i$ in it.\\

We thus see that $\matrixsym{C^{-1}}\matrixsym{\Psi}$ creates a new matrix $\matrixsym{\Phi}$ in the $\vec{\phi}$ basis.\\

Finally we note that\\

\begin{equation} \Phi=det(\matrixsym{\Psi}) \end{equation}

At this point we note that, taking advantage of the unitarity of $\matrixsym{C}$\\

\begin{equation} 
det(\matrixsym{\Phi})=det(\matrixsym{C^{-1}}\matrixsym{\Psi})=det(\matrixsym{C^{-1}})\Phi
\end{equation}

Now we then note that \\
\begin{equation}
det(\matrixsym{C^{-1}})=det({\matrixsym{C}^{T}})=|det({\matrixsym{C}})|
\end{equation}

From this it follows that 
\begin{equation}
det(\matrixsym{C^{-1}})=e^{i\theta}
\end{equation}

And so we have from here\\

\begin{equation}
det(\matrixsym{\Phi})=det(\matrixsym{C^{-1}}\matrixsym{\Phi})=e^{i\theta}\matrixsym{\Psi}
\end{equation}

And so we see that we can continue to write the slater determinant as a determinant in another basis.

\section{Variation Of Energy}

We now consider how energy might vary as a function of $\bra{\psi}$.\\

We then have that\\

\begin{equation}
\braket{E}=\bra{\psi+\delta\psi}H\ket{\psi}
\end{equation}

From this it follows that \\

\begin{equation}
\delta \braket{E}=\bra{\delta\psi}H\ket{\psi}
\end{equation}

Now, knowing\\

\begin{equation}
H=\sum_{i=1}^A(t(x_i)+u(x_i))+\frac{1}{2} \sum_{i\neq j}^A v(x_i,x_j)
\end{equation}

Hence we have that\\

\begin{equation}
\delta \braket{E}=\bra{\delta\psi}\left(\sum_{i=1}^A(t(x_i)+u(x_i))+\frac{1}{2} \sum_{i\neq j}^A v(x_i,x_j)\right)\ket{\psi}
\end{equation}
Now, we take advantage of our previous results for the expectation values of one and two body operators:

\begin{equation}
\delta \braket{E}=\sum_{i} \bra{\delta\psi_i}\left(t(x_i)+u(x_i)\right)\ket{\psi_i}+\sum_{i\neq j}^N\left( \bra{\delta \psi_i\psi_j}v\ket{\psi_i \psi_j} -\bra{\delta \psi_i\psi_j}v\ket{\psi_j \psi_i}  \right)
\end{equation}



\end{document}
