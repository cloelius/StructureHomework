% !TEX TS-program = pdflatex
% !TEX encoding = UTF-8 Unicode

% This is a simple template for a LaTeX document using the "article" class.
% See "book", "report", "letter" for other types of document.

\documentclass[11pt]{article} % use larger type; default would be 10pt

\usepackage[utf8]{inputenc} % set input encoding (not needed with XeLaTeX)

%%% Examples of Article customizations
% These packages are optional, depending whether you want the features they provide.
% See the LaTeX Companion or other references for full information.

%%% PAGE DIMENSIONS
\usepackage{geometry} % to change the page dimensions
\geometry{a4paper} % or letterpaper (US) or a5paper or....
% \geometry{margin=2in} % for example, change the margins to 2 inches all round
% \geometry{landscape} % set up the page for landscape
%   read geometry.pdf for detailed page layout information

\usepackage{graphicx} % support the \includegraphics command and options

% \usepackage[parfill]{parskip} % Activate to begin paragraphs with an empty line rather than an indent

%%% PACKAGES
\usepackage{booktabs} % for much better looking tables
\usepackage{array} % for better arrays (eg matrices) in maths
\usepackage{paralist} % very flexible & customisable lists (eg. enumerate/itemize, etc.)
\usepackage{verbatim} % adds environment for commenting out blocks of text & for better verbatim
\usepackage{subfig} % make it possible to include more than one captioned figure/table in a single float
% These packages are all incorporated in the memoir class to one degree or another...
\usepackage{isomath}
%%% HEADERS & FOOTERS
\usepackage{fancyhdr} % This should be set AFTER setting up the page geometry
\pagestyle{fancy} % options: empty , plain , fancy
\renewcommand{\headrulewidth}{0pt} % customise the layout...
\lhead{}\chead{}\rhead{}
\lfoot{}\cfoot{\thepage}\rfoot{}
\usepackage{braket}
%%% SECTION TITLE APPEARANCE
\usepackage{sectsty}
\allsectionsfont{\sffamily\mdseries\upshape} % (See the fntguide.pdf for font help)
% (This matches ConTeXt defaults)
\usepackage{amsmath}

%%% ToC (table of contents) APPEARANCE
\usepackage[nottoc,notlof,notlot]{tocbibind} % Put the bibliography in the ToC
\usepackage[titles,subfigure]{tocloft} % Alter the style of the Table of Contents
\renewcommand{\cftsecfont}{\rmfamily\mdseries\upshape}
\renewcommand{\cftsecpagefont}{\rmfamily\mdseries\upshape} % No bold!

%%% END Article customizations

%%% The "real" document content comes below...

\title{Exercise 3}
\author{Charles loelius}
%\date{} % Activate to display a given date or no date (if empty),
         % otherwise the current date is printed 

\begin{document}
\maketitle

\section{How many Slater Determinants are possible}

We have a three level system, with each level doubly degenerate. Given two particles, there are thus the equivalent of 6 independent states, each of which could have one single particle state. Hence there are ${ 6 \choose 2}=15$ possible two body states. We thus have that there must be 15 possible slater determinants. 

\section{Properties of Hamiltonian}
For this section we restrict our consideration only to the p=1,2 states.\\

We now consider a hamiltonian of the form\\
\begin{equation}
\hat{h_0}\psi_{p \sigma}=d\times p\psi_{p \sigma}
\end{equation}

We also know that $h_I$ merely has a constant value of -g between any two particle states, including between the same state. The total hamiltonian is of course then just \\
\begin{equation}
H=h_o+h_I
\end{equation}

Now given this, we make one final assumption, which is that there will only be two slater determinants in play, namely those where the two particles are in the same level but different spins. Thus we can mark $\ket{\psi_1}, \ket{\psi_2}$ for each possibility, where both particles are in d=1 or d=2 respectively. Letting our basis then be\\

\begin{equation}
\left( \begin{matrix} \ket{\psi_1} \\ \ket{\psi_2} \end{matrix} \right)
\end{equation}

It follows that, as there are two particles in each state \\
\begin{equation}
\hat{h_0}=\left(\begin{matrix} 2p & 0\\ 0 & 4p \end{matrix} \right)
\end{equation}
\begin{equation}
h_I=\left( \begin{matrix}-g & -g \\-g & -g \end{matrix}\right)
\end{equation}
We thus have that the overall hamiltonian must be\\

\begin{equation}
H=h_0+h_I=\left(\begin{matrix} 2p-g & -g \\ -g & 4p-g\end{matrix}\right)
\end{equation}

We can then trivially find the eigenvectors and the eigenvalues of this matrix in the normal fashion, setting the determinant to zero with a matrix $x\matrixsym{I}$ subtracted from it.\\

\begin{equation}
\left|\begin{matrix} 2p-g-x & -g \\ -g & 4p-g-x\end{matrix}\right|=0
\end{equation}

\begin{equation}
(2p-g-x)(4p-g-x)-g^2=0
\end{equation}
\begin{equation}
8p^2-2pg-2px-4pg+g^2+gx-4px+xg+x^2=x^2+x(-6p+2g)+8p^2-6pg=0
\end{equation}

\begin{equation}
x=(3p-g)\pm \sqrt{g^2+p^2}
\end{equation}

So the eigenvalues are\\

\begin{equation}
\epsilon_1=(3p-g)+\sqrt{g^2+p^2}
\end{equation}
\begin{equation}
\epsilon_2=(3p-g)-\sqrt{g^2+p^2}
\end{equation}
We then can find the eigenvectors as:\\

\begin{equation}
\chi_1 =\left( \begin{matrix} -g \\ p+\sqrt{g^2+p^2}\end{matrix}\right)
\end{equation}
\begin{equation}
\chi_2 =\left( \begin{matrix} -g \\ p-\sqrt{g^2+p^2}\end{matrix}\right)
\end{equation}

We thus see that in the energy eigenstates there is a mixing of the p=2 state in the p=1 state of $-\frac{p\pm\sqrt{g^2+p^2}}{g}$

We see tha as p goes to zero this becomes close to 1, suggesting equal amounts of both states(as is sensible, since they would be degenerate states), or if g goes to zero it becomes ill definied(since of course in this case the eigenstates would revert to the basis states). That is, these eigenvectors more accurately represent the slater determinants in the form:\\

\begin{equation}
\chi_x=\left( \begin {matrix} a & b \end{matrix} \right)\left(\begin{matrix} \psi_1 \\ \psi_2 \end{matrix} \right)=a\ket{\psi_1}+b\ket{\psi_2}
\end{equation}

Where a and b are (normalized) constants defining the eigenvector of the corresponding matrix and the psi's are of course the slater determinants from part a.







\end{document}
