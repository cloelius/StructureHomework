% !TEX TS-program = pdflatex
% !TEX encoding = UTF-8 Unicode

% This is a simple template for a LaTeX document using the "article" class.
% See "book", "report", "letter" for other types of document.

\documentclass[11pt]{article} % use larger type; default would be 10pt

\usepackage[utf8]{inputenc} % set input encoding (not needed with XeLaTeX)

%%% Examples of Article customizations
% These packages are optional, depending whether you want the features they provide.
% See the LaTeX Companion or other references for full information.

%%% PAGE DIMENSIONS
\usepackage{geometry} % to change the page dimensions
\geometry{a4paper} % or letterpaper (US) or a5paper or....
% \geometry{margin=2in} % for example, change the margins to 2 inches all round
% \geometry{landscape} % set up the page for landscape
%   read geometry.pdf for detailed page layout information

\usepackage{graphicx} % support the \includegraphics command and options

% \usepackage[parfill]{parskip} % Activate to begin paragraphs with an empty line rather than an indent

%%% PACKAGES
\usepackage{booktabs} % for much better looking tables
\usepackage{array} % for better arrays (eg matrices) in maths
\usepackage{paralist} % very flexible & customisable lists (eg. enumerate/itemize, etc.)
\usepackage{verbatim} % adds environment for commenting out blocks of text & for better verbatim
\usepackage{subfig} % make it possible to include more than one captioned figure/table in a single float
% These packages are all incorporated in the memoir class to one degree or another...

%%% HEADERS & FOOTERS
\usepackage{fancyhdr} % This should be set AFTER setting up the page geometry
\pagestyle{fancy} % options: empty , plain , fancy
\renewcommand{\headrulewidth}{0pt} % customise the layout...
\lhead{}\chead{}\rhead{}
\lfoot{}\cfoot{\thepage}\rfoot{}

%%% SECTION TITLE APPEARANCE
\usepackage{sectsty}
\usepackage{braket}
\allsectionsfont{\sffamily\mdseries\upshape} % (See the fntguide.pdf for font help)
% (This matches ConTeXt defaults)

%%% ToC (table of contents) APPEARANCE
\usepackage[nottoc,notlof,notlot]{tocbibind} % Put the bibliography in the ToC
\usepackage[titles,subfigure]{tocloft} % Alter the style of the Table of Contents
\renewcommand{\cftsecfont}{\rmfamily\mdseries\upshape}
\renewcommand{\cftsecpagefont}{\rmfamily\mdseries\upshape} % No bold!

%%% END Article customizations

%%% The "real" document content comes below...

\title{Projeect 2}
\author{Charles Loelius}
%\date{} % Activate to display a given date or no date (if empty),
         % otherwise the current date is printed 

\begin{document}
\maketitle

\section{Hamiltonian for Pair Conservation}

In this problem we consider a space of equally spaced single particle levels with a spin degeneracy(up and down spin). We can then consider a pair conserving Hamiltonian with a single particle part and a pairing term. We represent these as\\

\begin{equation}
H=H_0+V
\end{equation}

Where\\

\begin{equation}
H_0=\chi \sum_{p\sigma} \left(p-1\right)a_{p\sigma}^\dagger a_{p\sigma}
\end{equation}  

\begin{equation}
\braket{q_+q_-|V|s_+s_-} =-g \rightarrow V=-g\sum_{pq} a_{+p}^\dagger a_{p-}^\dagger a_{q+}a_{q-}
\end{equation}

Finally we define a pair creation and annhilation operator as\\
\begin{equation}\hat{P_p}=a_{p+} a_{p-}\end{equation}
\begin{equation}\hat{P_p}^\dagger=a_{p+}^\dagger a_{p-}^\dagger\end{equation}

From this it follows that\\

Now let us consider each of the commutation relations between the single particle and creation annhilation operators.\\

\begin{equation}
\left[P_p, P_q\right]=0
\end{equation}
\begin{equation}
\left[P_p,P_q^\dagger\right]=\delta_{p,q}
\end{equation}
\begin{equation}
\left[P_p,a_{q\sigma}\right]=0
\end{equation}
\begin{equation}
\left[P_p^\dagger,a_{q\sigma}\right]=-\delta_{p,q}a_{p-\sigma}^\dagger
\end{equation}
\begin{equation}
\left[P_p,a_{q\sigma}^\dagger \right]=\delta_{p,q}a_{p -\sigma}\end{equation}
\begin{equation}
\left[P_p^\dagger,a_{q \sigma}^\dagger \right]=0\end{equation}
\begin{equation}
\left[P_q^\dagger,P_p^\dagger\right]=0\end{equation}
We note that we can write V as \\
\begin{equation}
V=-g\sum_{pq}P_q^\dagger P_p \end{equation}



\subsection{$H_0,V$ commute with$S_z, S^2$}
\subsubsection{$H_0$ commutes with $S_z$}
\begin{equation}
\left[H_0,S_z\right]=\epsilon \sum_{p' \sigma' p\sigma} \left(p-1\right) \sigma' \left[a_{p\sigma}^\dagger a_{p\sigma},a_{p'\sigma'}^\dagger a_{p'\sigma'}\right]=A\left(\delta_{p,p'}\delta_{\sigma,\sigma'}-\delta_{p,p'}\delta_{\sigma,\sigma'}\right)=0\end{equation}
\subsubsection{$V$ commutes with $S_z$}
\begin{equation}
\left[V,S_z\right]=A\sum_{p,q,p',\sigma}\left[P_p^\dagger P_q,a_{p'\sigma}^\dagger a_{p\sigma} \right]=
\end{equation}\begin{equation}A\sum_{p,q,p',\sigma}\left(P_{p}^\dagger\left[P_{p},a_{p'\sigma}^\dagger \right]a_{p'\sigma}+P_p^\dagger a_{p'\sigma}^\dagger \left[P_p,a_{p'\sigma}\right]+\left[P_p^\dagger a_{p'\sigma}^\dagger \right]P_p,a_{p'\sigma}+a_{p'\sigma}^\dagger \left[P_p^\dagger,a_{p'\sigma}\right]P_p\right)
\end{equation}

\begin{equation}
= A\sum_{p,q,p'\sigma} \left(\delta_{p,p'}P_p^\dagger a_{p'-\sigma}a_{p'\sigma}-\delta_{p,p'}a_{p',\sigma}^\dagger a^\dagger_{p,-\sigma}P_p\right)=A\sum_{p\sigma} \left(P_p^\dagger P_p-P_p^\dagger P_p\right)=0\end{equation}


\subsubsection{$H_0$ commutes with $S^2$}

We note that \\
\begin{equation}
S^2=S_z^2+\frac{1}{2}\left(S_+S_- +S_-S_+ \right)
\end{equation}

Knowing from above that $H_0$ commutes with $S_z$ it thereby follows that  we must only prove\\

\begin{equation}
\left[H_0,\left(S_+S_- +S_-S_+ \right)\right]=0
\end{equation}

We note for any operator $O$ \\

\begin{equation}
\left[O,\left(S_+S_- +S_-S_+ \right)\right] =\left[O,S_+\right]S_-+\left[O,S_-\right]S_+ + S_+\left[O,S_-\right]+S_-\left[O,S_+\right]
\end{equation}


Now we then note that this can be written as \\

\begin{equation}
\left[

\subsection{Hamiltonian Commutes with $P_p^+ P_p^-$}

In order to show it keeps pairs together we must show that it conserves the product of the pair creation and annhilation operators. 


Now we can then take advantage of some basic commutator algebra to solve that\\

\begin{equation}
\left[P_p^\dagger P_p,P_q^\dagger P_r \right]=P_p^\dagger \left[P_p,P_q^\dagger\right]P_r+P_q^\dagger P_q^\dagger \left[P_p,P_r \right] +\left[P_p^\dagger,P_q^\dagger\right]P_r P_p+P_q^\dagger \left[P_p^\dagger, P_r \right]P_p\end{equation}\begin{equation}=\delta_{p,q}P_p^\dagger P_r -\delta_{p,r}P_q^\dagger P_p=P_q^\dagger P_r - P_q^\dagger P_r=0 \end{equation}

\begin{equation}
\left[P_p^\dagger P_p, a_{r \sigma}^\dagger a_{r\sigma}\right]=P_p^\dagger \left[P_p,a_{r\sigma}^\dagger\right]a_{r\sigma}+
P_p^\dagger a_{r\sigma}^\dagger \left[P_p,a_{r \sigma}\right] + \left[P_p^\dagger, a_{r\sigma}^\dagger \right] a_{r \sigma} P_p + a_{r \sigma}^\dagger \left[P_p^\dagger, a_{r \sigma} \right] P_p \end{equation}




\begin{equation}
\left[P_p^\dagger P_p, a_{r \sigma}^\dagger a_{r\sigma}\right]=P_p^\dagger\delta_{p,r} a_{p -\sigma} a_{r\sigma}-\delta_{p,r}a_{r\sigma}^\dagger a_{p-\sigma}^\dagger P_p=P_p^\dagger P_p-P_p^\dagger P_p =0\end{equation}
%
%\begin{equation}
%\left[P_p^\dagger P_p ,V\right]=-g \sum_{ab} P_a^\dagger P_b P_p^\dagger P_p +gP_p^\dagger P_p\sum_{ab}P_a^\dagger P_b
%\end{equation}
%
%Now we note for all cases where $a,b \neq b$ it follows that the pair operators commute with the terms and so those terms cancel. Furthermore in the case where $a=b=p$ there is a trivial commutation. We are thus left with only the term where $a=p, b\neq p$ or $b=p, a \neq p$.\\
%
%In the first of the cases we thus have, using standard commutator algebra and remembering that $p\neq b$.\\
%
%\begin{equation}
%\left[P_p^\dagger P_p, P_p^\dagger P_b \right]= P_p^\dagger \left[P_p,P_p^\dagger\right]P_b +P_p^\dagger P_p^\dagger \left[P_p,P_b\right]+\left[P_p^\dagger,P_p^\dagger\right]P_p P_b+P_p^\dagger \left[P_p^\dagger,P_b\right] P_p=P_p^\dagger \left[P_p,P_p^\dagger\right]P_b\end{equation}
%
%And doing the same for the second we have\\
%\begin{equation}
%\left[P_p^\dagger P_p, P_a^\dagger P_p \right]= P_p^\dagger \left[P_p,P_a^\dagger\right]P_p +P_p^\dagger P_a^\dagger \left[P_p,P_p\right]+\left[P_a^\dagger,P_p^\dagger\right]P_p P_p+P_a^\dagger \left[P_p^\dagger,P_p\right] P_p=P_a^\dagger \left[P_p^\dagger,P_p\right]P_p=-P_a^\dagger \left[P_p,P_p^\dagger\right]P_p\end{equation}
%
%Finally we note that the commutator $\left[P_p^\dagger,P_p \right]$ is just 1 looking at the commutator of the individual creation and annhilation operators. As we are summing over all a and b that each of these variables is a dummy variable and so all of these commutators summed will go to zero. As such the Hamiltonian commutes with the pair operator. 

\end{document}
